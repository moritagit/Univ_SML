\documentclass[class=jsarticle, crop=false, dvipdfmx, fleqn]{standalone}
%% preamble for Numerical-structure-analysis report

\input{/Users/User/Documents/Project/TeX/preamble/mypreamble}

%% titles
\title{統計的機械学習 レポート}
\author{37-196360 \quad 森田涼介}


%% setting for listings
\newtcbinputlisting[auto counter]{\reportlisting}[3][]{%
	listing file = {#3},
	listing options = {language=python, style=tcblatex, numbers=left, numberstyle=\tiny},
	listing only,
	breakable,
	toprule at break = 0mm,
	bottomrule at break = 0mm,
	left = 6mm,
	sharp corners,
	drop shadow,
	title = Listings \thetcbcounter : \texttt{#2},
	label = #1,
	}



%% title format
\usepackage{titlesec}
\titleformat{\section}{\LARGE}{宿題\thesection}{0zw}{}
\newcommand{\sectionbreak}{\clearpage}
\titleformat{\subsection}{\Large}{\Alph{subsection})}{0zw}{}

\begin{document}
\section{}

原信号\(\qty{\bm{s}_i}_{i=1}^{n} \ \qty(\bm{s}_i \in \mathbb{R}^d)\),
観測信号\(\qty{\bm{x}_i}_{i=1}^{n} \ \qty(\bm{x}_i \in \mathbb{R}^d)\)
について,
混合行列\(\bm{M}\ (\in \mathbb{R}^{d \times d}) \)を用いて,
\begin{equation}
    \bm{x}_i = \bm{M} \bm{s}_i
    \label{eq:s2x}
\end{equation}
が成立するときを考える。
ここで,原信号\(\qty{\bm{s}_i}_{i=1}^{n}\)はi.i.d,
期待値\(\bm{\mu}_s = \bm{0}\)で,
共分散行列が単位行列\(\bm{\Sigma}_s = \bm{I}_d\)であり,
また,\(\bm{M}\)は逆行列を持つと仮定する(独立成分分析)。

観測信号を球状化(白色化)することを考える。
観測信号の期待値と共分散行列は次のようになる。
\begin{align}
    \bm{\mu}_x
        & = \frac{1}{n} \sum_{i=1}^{n} \bm{x}_i \\
        & = \frac{1}{n} \sum_{i=1}^{n} \bm{M} \bm{s}_i \\
        & = \bm{M} \frac{1}{n} \sum_{i=1}^{n} \bm{s}_i \\
        & = \bm{M} \bm{\mu}_s \\
        & = \bm{0}
\end{align}
\begin{align}
    \bm{\Sigma}_x
        & = \frac{1}{n} \sum_{i=1}^{n} (\bm{x}_i - \bm{\mu}_x)^\mathrm{T} (\bm{x}_i - \bm{\mu}_x) \\
        & = \frac{1}{n} \sum_{i=1}^{n} \bm{x}_i^\mathrm{T} \bm{x}_i \\
        & \equiv \bm{C}
\end{align}
これより,\(\bm{\Sigma}_x = \bm{C}\)が逆行列を持つことを仮定し,
\(\bm{x}_i\)を白色化すると,
\begin{align}
    \tilde{\bm{x}}_i
        & = \bm{\Sigma}^{-\frac{1}{2}} (\bm{x}_i - \bm{\mu}_x) \\
        & = \bm{C}^{-\frac{1}{2}} \bm{x}_i
\end{align}
また,式(\ref{eq:s2x})の両辺に左側から\(\bm{C}^{-1/2}\)をかけることで,
\begin{align}
    & \tilde{\bm{x}}_i = \tilde{\bm{M}} \bm{s}_i
        \label{eq:s2x_tilde} \\
    & \tilde{\bm{M}} = \bm{C}^{-\frac{1}{2}} \bm{M}
\end{align}
となる。

白色化された\(\tilde{\bm{x}}_i\)(平均\(\bm{0}\))の共分散行列を考える。
\(\bm{s}_i\)は平均\(\bm{0}\),共分散行列\(\bm{I}_d\)であることに注意しつつ,
式(\ref{eq:s2x_tilde})を用いれば,
\begin{align}
    \bm{\Sigma}_{\tilde{x}}
        & = \frac{1}{n} \sum_{i=1}^{n} \tilde{\bm{x}}_i \tilde{\bm{x}}_i^\mathrm{T} \\
        & = \frac{1}{n} \sum_{i=1}^{n} (\tilde{\bm{M}} \bm{s}_i) (\bm{s}_i^\mathrm{T} \tilde{\bm{M}}^\mathrm{T}) \\
        & = \tilde{\bm{M}} \qty(\frac{1}{n} \sum_{i=1}^{n} \bm{s}_i \bm{s}_i^\mathrm{T}) \tilde{\bm{M}}^\mathrm{T} \\
        & = \tilde{\bm{M}} \bm{\Sigma}_s \tilde{\bm{M}}^\mathrm{T} \\
        & = \tilde{\bm{M}} \bm{I}_d \tilde{\bm{M}}^\mathrm{T} \\
        & = \tilde{\bm{M}} \tilde{\bm{M}}^\mathrm{T}
\end{align}
白色化より,当然これは単位行列\(\bm{I}_d\)となる。
結局,
\begin{equation}
    \tilde{\bm{M}} \tilde{\bm{M}}^\mathrm{T} = \bm{I}_d
    \label{eq:result}
\end{equation}
を得る。
式(\ref{eq:result})は\(\tilde{\bm{M}}\)が直交行列であることを示している。
つまり,
\begin{equation}
    \tilde{\bm{M}}^{-1} = \tilde{\bm{M}}^\mathrm{T}
\end{equation}
が成立する。
これより,次式が成立する。
\begin{equation}
    \tilde{\bm{M}} \tilde{\bm{M}}^\mathrm{T} = \tilde{\bm{M}}^\mathrm{T} \tilde{\bm{M}} = \bm{I}_d
\end{equation}


\end{document}
