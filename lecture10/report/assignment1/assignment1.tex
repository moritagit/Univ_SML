\documentclass[class=jsarticle, crop=false, dvipdfmx, fleqn]{standalone}
%% preamble for Numerical-structure-analysis report

\input{/Users/User/Documents/Project/TeX/preamble/mypreamble}

%% titles
\title{統計的機械学習 レポート}
\author{37-196360 \quad 森田涼介}


%% setting for listings
\newtcbinputlisting[auto counter]{\reportlisting}[3][]{%
	listing file = {#3},
	listing options = {language=python, style=tcblatex, numbers=left, numberstyle=\tiny},
	listing only,
	breakable,
	toprule at break = 0mm,
	bottomrule at break = 0mm,
	left = 6mm,
	sharp corners,
	drop shadow,
	title = Listings \thetcbcounter : \texttt{#2},
	label = #1,
	}



%% title format
\usepackage{titlesec}
\titleformat{\section}{\LARGE}{宿題\thesection}{0zw}{}
\newcommand{\sectionbreak}{\clearpage}
\titleformat{\subsection}{\Large}{\Alph{subsection})}{0zw}{}

\begin{document}

\section*{宿題1}

\(q(\bm{u}_j)\)について次の比例式が成立する。
\begin{equation}
    q(\bm{u}_j) \propto p(\bm{u}_j | \bm{\Lambda}_u) \exp{\sum_{j,\ i \in \mathcal{O}} \int q(\bm{v}_i) \log{p(r_{j,\ i} | \bm{u}_j^\mathrm{T} \bm{v}_i,\ \sigma^2)} \dd{\bm{v}_i}}
    \label{eq:q_u_margin}
\end{equation}
ここで,\(p(\bm{u}_j | \bm{\Lambda}_u)\)について,
\begin{equation}
    p(\bm{u}_j | \bm{\Lambda}_u) \propto \exp(- \frac{1}{2} \bm{u}_j^\mathrm{T} \bm{\Lambda}_u \bm{u}_j)
    \label{eq:p_u_cond}
\end{equation}
が成立する。
また,
\begin{align}
    l_{u,\ i,\ j}
        & \equiv \int q(\bm{v}_i) \log{p(r_{j,\ i} | \bm{u}_j^\mathrm{T} \bm{v}_i,\ \sigma^2)} \dd{\bm{v}_i} \\
        & = - \frac{1}{2 \sigma^2} \int q(\bm{v}_i) (r_{j,\ i} - \bm{u}_j^\mathrm{T} \bm{v}_i)^2 \dd{\bm{v}_i} \notag \\
        & = - \frac{1}{2 \sigma^2} \int q(\bm{v}_i) \qty(r_{j,\ i}^2 - 2 r_{j,\ i} \bm{u}_j^\mathrm{T} \bm{v}_i + \bm{u}_j^\mathrm{T} \bm{v}_i \bm{v}_i^\mathrm{T} \bm{u}_j) \dd{\bm{v}_i} \notag \\
        & = - \frac{1}{2 \sigma^2} \qty(r_{j,\ i}^2 - 2 r_{j,\ i} \bm{u}_j^\mathrm{T} \mathbb{E}_{q(\bm{v}_i)} \qty[\bm{v}_i] + \bm{u}_j^\mathrm{T} \mathbb{E}_{q(\bm{v}_i)} \qty[\bm{v}_i \bm{v}_i^\mathrm{T}] \bm{u}_j)
        \label{eq:l_u}
\end{align}
となる。
式(\ref{eq:q_u_margin}),(\ref{eq:p_u_cond}),(\ref{eq:l_u})から,
\begin{align}
    q(\bm{u}_j)
        & \propto \exp(- \frac{1}{2} \bm{u}_j^\mathrm{T} \bm{\Lambda}_u \bm{u}_j) \exp{\sum_{j,\ i \in \mathcal{O}} l_{u,\ i,\ j}} \notag \\
        & = \exp(- \frac{1}{2} \bm{u}_j^\mathrm{T} \bm{\Lambda}_u \bm{u}_j) \exp{\sum_{j,\ i \in \mathcal{O}} - \frac{1}{2 \sigma^2} \qty(r_{j,\ i}^2 - 2 r_{j,\ i} \bm{u}_j^\mathrm{T} \mathbb{E}_{q(\bm{v}_i)} \qty[\bm{v}_i] + \bm{u}_j^\mathrm{T} \mathbb{E}_{q(\bm{v}_i)} \qty[\bm{v}_i \bm{v}_i^\mathrm{T}] \bm{u}_j)} \notag \\
        & = \exp{- \frac{1}{2 \sigma^2} \qty(-2\bm{u}_j \qty(\sum_{j,\ i \in \mathcal{O}} r_{j,\ i} \mathbb{E}_{q(\bm{v}_i)} \qty[\bm{v}_i]) + \bm{u}_j^\mathrm{T} \qty(\sum_{j,\ i \in \mathcal{O}} \mathbb{E}_{q(\bm{v}_i)} \qty[\bm{v}_i \bm{v}_i^\mathrm{T}] + \sigma^2 \bm{\Lambda}_u) \bm{u}_j)}
        \label{eq:q_u}
\end{align}
いま,多次元正規分布に従う確率変数\(\bm{x}\)について次が成立する。
\begin{equation}
    q(\bm{x})
        = \mathcal{N}(\bm{x} | \bm{\mu},\ \bm{\Sigma})
        \propto \exp{-\frac{1}{2} \qty(\bm{x}^\mathrm{T} \bm{\Sigma}^{-1} \bm{x} - 2\bm{x} \bm{\Sigma}^{-1} \bm{\mu})}
    \label{eq:multi_normal}
\end{equation}
これと式(\ref{eq:q_u})を比べると,
\begin{align}
    & q(\bm{u}_j) = \mathcal{N}(\bm{u}_j | \bm{\mu}_{u,\ j},\ \bm{V}_{u,\ j}) \\
    & \bm{V}_{u,\ j}^{-1} \bm{\mu}_{u,\ j} = \frac{1}{\sigma^2} \sum_{j,\ i \in \mathcal{O}} r_{j,\ i} \mathbb{E}_{q(\bm{v}_i)} \qty[\bm{v}_i] \\
    & \bm{V}_{u,\ j}^{-1} = \frac{1}{\sigma^2} \qty(\sum_{j,\ i \in \mathcal{O}} \mathbb{E}_{q(\bm{v}_i)} \qty[\bm{v}_i \bm{v}_i^\mathrm{T}] + \sigma^2 \bm{\Lambda}_u)
\end{align}
となることがわかる。
これを整理すると,
\begin{align}
    & q(\bm{u}_j) = \mathcal{N}(\bm{u}_j | \bm{\mu}_{u,\ j},\ \bm{V}_{u,\ j}) \\
    & \bm{\mu}_{u,\ j} = \frac{1}{\sigma^2} \bm{V}_{u,\ j} \sum_{j,\ i \in \mathcal{O}} r_{j,\ i} \mathbb{E}_{q(\bm{v}_i)} \qty[\bm{v}_i] \\
    & \bm{V}_{u,\ j} = \sigma^2 \qty(\sum_{j,\ i \in \mathcal{O}} \mathbb{E}_{q(\bm{v}_i)} \qty[\bm{v}_i \bm{v}_i^\mathrm{T}] + \sigma^2 \bm{\Lambda}_u)^{-1}
\end{align}
となる。

同様に,\(\bm{v}_i\)について,
\begin{align}
    q(\bm{v}_i)
        & \propto p(\bm{v}_i | \bm{\Lambda}_v) \exp{\sum_{j,\ i \in \mathcal{O}} \int q(\bm{u}_j) \log{p(r_{j,\ i} | \bm{u}_j^\mathrm{T} \bm{v}_i,\ \sigma^2)} \dd{\bm{u}_j}}
        \label{eq:q_v_margin} \\
        & = \exp(- \frac{1}{2} \bm{v}_i^\mathrm{T} \bm{\Lambda}_v \bm{v}_i) \exp{\sum_{j,\ i \in \mathcal{O}} - \frac{1}{2 \sigma^2} \qty(r_{j,\ i}^2 - 2 r_{j,\ i} \bm{v}_i^\mathrm{T} \mathbb{E}_{q(\bm{u}_j)} \qty[\bm{u}_j] + \bm{v}_i^\mathrm{T} \mathbb{E}_{q(\bm{u}_j)} \qty[\bm{u}_j \bm{u}_j^\mathrm{T}] \bm{v}_i)} \notag \\
        & = \exp{- \frac{1}{2 \sigma^2} \qty(-2\bm{v}_i \qty(\sum_{j,\ i \in \mathcal{O}} r_{j,\ i} \mathbb{E}_{q(\bm{u}_j)} \qty[\bm{u}_j]) + \bm{v}_i^\mathrm{T} \qty(\sum_{j,\ i \in \mathcal{O}} \mathbb{E}_{q(\bm{u}_j)} \qty[\bm{u}_j \bm{u}_j^\mathrm{T}] + \sigma^2 \bm{\Lambda}_v) \bm{v}_i)}
        \label{eq:q_v}
\end{align}
となることから,
\begin{align}
    & q(\bm{v}_i) = \mathcal{N}(\bm{v}_i | \bm{\mu}_{v,\ i},\ \bm{V}_{v,\ i}) \\
    & \bm{V}_{v,\ i}^{-1} \bm{\mu}_{v,\ i} = \frac{1}{\sigma^2} \sum_{j,\ i \in \mathcal{O}} r_{j,\ i} \mathbb{E}_{q(\bm{u}_j)} \qty[\bm{u}_j] \\
    & \bm{V}_{v,\ i}^{-1} = \frac{1}{\sigma^2} \qty(\sum_{j,\ i \in \mathcal{O}} \mathbb{E}_{q(\bm{u}_j)} \qty[\bm{u}_j \bm{u}_j^\mathrm{T}] + \sigma^2 \bm{\Lambda}_v)
\end{align}
を得て,結局,
\begin{align}
    & q(\bm{v}_i) = \mathcal{N}(\bm{v}_i | \bm{\mu}_{v,\ i},\ \bm{V}_{v,\ i}) \\
    & \bm{\mu}_{v,\ i} = \frac{1}{\sigma^2} \bm{V}_{v,\ i} \sum_{j,\ i \in \mathcal{O}} r_{j,\ i} \mathbb{E}_{q(\bm{u}_j)} \qty[\bm{u}_j] \\
    & \bm{V}_{v,\ i} = \sigma^2 \qty(\sum_{j,\ i \in \mathcal{O}} \mathbb{E}_{q(\bm{u}_j)} \qty[\bm{u}_j \bm{u}_j^\mathrm{T}] + \sigma^2 \bm{\Lambda}_v)^{-1}
\end{align}
を得る。



\end{document}
