\documentclass[class=jsarticle, crop=false, dvipdfmx, fleqn]{standalone}
\input{../../../preamble}
\begin{document}
\section{}

線形判別分析に基づき0--9までの10クラスの手書き文字認識を行う。
プログラムは\pageref{listing:assignment4}ページのlisting\ref{listing:assignment4}に示した。

以下に結果を示す。
混同行列は表\ref{tab:confusion_matrix}のようになり,
また,各カテゴリごとの正解率等は表\ref{tab:result}のようになった。

なお,マハラノビス距離に基づく分類を行ってみたところ,
trainデータの分散行列の行列式が10となり逆行列が計算できなかったため,
こちらは断念した。

\begin{table}[H]
	\centering
	\caption{混同行列}
	\begin{tabular}{|c||cccccccccc|} \hline
			& 0 & 1 & 2 & 3 & 4 & 5 & 6 & 7 & 8 & 9 \\ \hline\hline
		0	& 192 & 0 & 0 & 3 & 0 & 0 & 4 & 0 & 1 & 0 \\
		1	& 0 & 199 & 0 & 0 & 0 & 1 & 0 & 0 & 0 & 0 \\
		2	& 0 & 0 & 169 & 8 & 8 & 1 & 2 & 4 & 8 & 0 \\
		3	& 1 & 0 & 0 & 182 & 1 & 5 & 0 & 2 & 8 & 1 \\
		4	& 0 & 2 & 2 & 0 & 182 & 0 & 1 & 0 & 3 & 10 \\
		5	& 4 & 0 & 0 & 21 & 4 & 162 & 1 & 0 & 4 & 4 \\
		6	& 3 & 1 & 2 & 0 & 1 & 5 & 185 & 0 & 3 & 0 \\
		7	& 1 & 2 & 0 & 1 & 5 & 1 & 0 & 181 & 0 & 9 \\
		8	& 3 & 0 & 1 & 16 & 6 & 6 & 0 & 1 & 164 & 3 \\
		9	& 0 & 1 & 0 & 0 & 8 & 0 & 0 & 7 & 2 & 182 \\
		\hline
	\end{tabular}
	\label{tab:confusion_matrix}
\end{table}

\begin{table}[H]
	\centering
	\caption{各カテゴリごとの結果}
	\begin{tabular}{lrrr}
		Category & \# Data & \# Correct & Accuracy \\ \hline
		0 & 200 & 192 & 0.960 \\
		1 & 200 & 199 & 0.995 \\
		2 & 200 & 169 & 0.845 \\
		3 & 200 & 182 & 0.910 \\
		4 & 200 & 182 & 0.910 \\
		5 & 200 & 162 & 0.810 \\
		6 & 200 & 185 & 0.925 \\
		7 & 200 & 181 & 0.905 \\
		8 & 200 & 164 & 0.820 \\
		9 & 200 & 182 & 0.910
	\end{tabular}
	\label{tab:result}
\end{table}



\end{document}
