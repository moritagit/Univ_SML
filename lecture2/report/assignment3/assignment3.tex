\documentclass[class=jsarticle, crop=false, dvipdfmx, fleqn]{standalone}
\input{../../../preamble}
\begin{document}
\section{}


各カテゴリの分散共分散行列が等しいので,対数事後確率は以下のように表される。
\begin{equation}
	\log p(y|x) = {\mu_y}^\mathrm{T} \Sigma^{-1} x - \frac{1}{2} {\mu_y}^\mathrm{T} \Sigma^{-1} \mu_y + \log p_y + C''
\end{equation}
いま,2値分類問題を考えるから,
ある\(x\)について\(\log p(y=1|x) > \log p(y=2|x)\)となるときに\(x\)はカテゴリ\(\hat{y}=1\)に,
\(\log p(y=1|x) < \log p(y=2|x)\)となるときに\(x\)はカテゴリ\(\hat{y}=2\)に分類される。

これを実装した結果が\pageref{listing:assignment3}ページのlisting \ref{listing:assignment3}である。
この実行結果より,\(x_1\)と\(x_2\)は,それぞれ100{\%}の正解率で分類できることがわかる。


\end{document}