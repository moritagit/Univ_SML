\documentclass[class=jsarticle, crop=false, dvipdfmx, fleqn]{standalone}
\input{../../../preamble}
\begin{document}
\section{}

入力次元\(d = 2\),カテゴリ数\(c = 2\),
各カテゴリの事前分布\(p(y=1) = p(y=2) = 1/2 \)の分類問題を考える。
また,各カテゴリの条件付き確率\(p(x|y)\)は正規分布であるとし,
その期待値と共分散行列は次のようになるとする。
\begin{align}
	& \mu_1 = \begin{bmatrix} 2 \\ 0 \end{bmatrix}, \qquad
	\mu_2 = \begin{bmatrix} -2 \\ 0 \end{bmatrix} \\
	& \Sigma_1 = \Sigma_2 = \Sigma = 
		\begin{bmatrix}
			9-8{\cos}^2 \beta & 8 \sin\beta \cos\beta \\
			8 \sin\beta \cos\beta & 9-8{\sin}^2 \beta
		\end{bmatrix}
\end{align}

線形判別分析に基づいて決定境界を求める。
決定境界は,
\begin{equation}
	p(y=1|x) = p(y=2|x)
	\label{eq:decision_boundary}
\end{equation}
で与えられる。
ここで,各カテゴリの分散共分散行列が等しいので,
\begin{equation}
	\log\ p(y|x) = {\mu_y}^\mathrm{T} \Sigma^{-1} x - \frac{1}{2} {\mu_y}^\mathrm{T} \Sigma^{-1} \mu_y + \log{p_y} + C''
	\label{eq:log_probability}
\end{equation}
となる。よって
\begin{align}
	& \log p(y=1|x) = {\mu_1}^\mathrm{T} \Sigma^{-1} x - \frac{1}{2} {\mu_1}^\mathrm{T} \Sigma^{-1} \mu_1 + \log{p_1} + C'' \\
	& \log p(y=2|x) = {\mu_2}^\mathrm{T} \Sigma^{-1} x - \frac{1}{2} {\mu_2}^\mathrm{T} \Sigma^{-1} \mu_2 + \log{p_2} + C''
\end{align}
式(\ref{eq:decision_boundary})と\(p(y=1) = p(y=2) = 1/2 \)とから,
辺々引いて,
\begin{equation}
	({\mu_1}^\mathrm{T} - {\mu_2}^\mathrm{T}) \Sigma^{-1} x - \frac{1}{2} ({\mu_1}^\mathrm{T} \Sigma^{-1} \mu_1 - {\mu_2}^\mathrm{T} \Sigma^{-1} \mu_2) = 0
	\label{eq:answer_character_expression}
\end{equation}
を得る。
いま,\(\Sigma\)の逆行列を求めると,
\begin{align}
	& \Sigma^{-1} = \frac{1}{9}
		\begin{bmatrix}
			9-8{\sin}^2 \beta & -8 \cos\beta \sin\beta \\
			-8 \cos\beta \sin\beta & 9-8{\cos}^2 \beta
		\end{bmatrix}
\end{align}
となるので,
\begin{align}
	({\mu_1}^\mathrm{T} - {\mu_2}^\mathrm{T}) \Sigma^{-1}
		& =
		\begin{bmatrix} 4 & 0 \end{bmatrix} \cdot \frac{1}{9}
		\begin{bmatrix}
			9-8{\sin}^2 \beta & -8 \cos\beta \sin\beta \\
			-8 \cos\beta \sin\beta & 9-8{\cos}^2 \beta
		\end{bmatrix} \\
		& = \frac{4}{9} \begin{bmatrix} 9-8{\sin}^2 \beta & -8 \cos\beta \sin\beta \end{bmatrix} \\
	{\mu_1}^\mathrm{T} \Sigma^{-1} \mu_1
		& = 
		\begin{bmatrix} 2 & 0 \end{bmatrix} \cdot \frac{1}{9}
		\begin{bmatrix}
			9-8{\sin}^2 \beta & -8 \cos\beta \sin\beta \\
			-8 \cos\beta \sin\beta & 9-8{\cos}^2 \beta
		\end{bmatrix}
		\cdot \begin{bmatrix} 2 \\ 0 \end{bmatrix}
		= \frac{4}{9} (9-8{\sin}^2 \beta) \\
	{\mu_2}^\mathrm{T} \Sigma^{-1} \mu_2
		& = 
		\begin{bmatrix} -2 & 0 \end{bmatrix} \cdot \frac{1}{9}
		\begin{bmatrix}
			9-8{\sin}^2 \beta & -8 \cos\beta \sin\beta \\
			-8 \cos\beta \sin\beta & 9-8{\cos}^2 \beta
		\end{bmatrix}
		\cdot \begin{bmatrix} -2 \\ 0 \end{bmatrix}
		= \frac{4}{9} (9-8{\sin}^2 \beta)
\end{align}
となる。
これらを式(\ref{eq:answer_character_expression})に代入して,
\begin{align}
	& \frac{4}{9} \qty((9-8{\sin}^2 \beta) x^{(1)} -8 \cos\beta \sin\beta x^{(2)}) -\frac{1}{2} \qty(\frac{4}{9} (9-8{\sin}^2 \beta) - \frac{4}{9} (9-8{\sin}^2 \beta)) = 0 \\
	& (9-8{\sin}^2 \beta) x^{(1)} -8 \cos\beta \sin\beta x^{(2)} = 0 \\
	& x^{(1)} = -\frac{8 \cos\beta \sin\beta}{9-8{\sin}^2 \beta} x^{(2)}
\end{align}
よって,決定境界は,
\begin{equation}
	x^{(1)} = -\frac{8 \cos\beta \sin\beta}{9-8{\sin}^2 \beta} x^{(2)}
\end{equation}
である。

\end{document}