\documentclass[dvipdfmx, fleqn]{jsarticle}
%% preamble for Numerical-structure-analysis report

\input{/Users/User/Documents/Project/TeX/preamble/mypreamble}

%% titles
\title{統計的機械学習 レポート}
\author{37-196360 \quad 森田涼介}


%% setting for listings
\newtcbinputlisting[auto counter]{\reportlisting}[3][]{%
	listing file = {#3},
	listing options = {language=python, style=tcblatex, numbers=left, numberstyle=\tiny},
	listing only,
	breakable,
	toprule at break = 0mm,
	bottomrule at break = 0mm,
	left = 6mm,
	sharp corners,
	drop shadow,
	title = Listings \thetcbcounter : \texttt{#2},
	label = #1,
	}



%% title format
\usepackage{titlesec}
\titleformat{\section}{\LARGE}{宿題\thesection}{0zw}{}
\newcommand{\sectionbreak}{\clearpage}
\titleformat{\subsection}{\Large}{\Alph{subsection})}{0zw}{}

\title{
	統計的機械学習 \\
	第九回 レポート ID: 01
	}
\author{37-196360 \quad 森田涼介}
\begin{document}
\maketitle



あるデータ\(x_i\)について,
その潜在変数を\(z_i\)とし,
また,事前分布のパラメータを\(\theta\)と表す。
Jensenの不等式を用いることで,
次のように変分下限\(L\)が求まる。
\begin{align}
    \log{p_{\theta} (x_i)}
        & = \log{\int p_{\theta} (x_i,\ z_i) \dd{z}}
            \qquad (\text{潜在変数\(z_i\)は観測できないため周辺化}) \\
        & = \log{\int q(z_i) \frac{p_{\theta} (x_i,\ z_i)}{q(z_i)} \dd{z}} \\
        & = \log{E_{q(z_i)} \qty[\frac{p_{\theta} (x_i,\ z_i)}{q(z_i)}]} \\
        & \ge E_{q(z_i)} \qty[\log{\frac{p_{\theta} (x_i,\ z_i)}{q(z_i)}}]
            \qquad (\because \text{Jensenの不等式}) \\
        & = \int q(z_i) \log{\frac{p_{\theta} (x_i,\ z_i)}{q(z_i)}} \dd{z_i} \\
        & = \int q(z_i) \log{\frac{p_{\theta} (x_i | z_i) p(z_i)}{q(z_i)}} \dd{z_i} \\
        & \equiv L\qty[q(z_i),\ \theta;\ x_i]
\end{align}
これを用いると,
複数のデータ\(x_{1:n}\)があるとき,
その生成確率の対数尤度と変分下限は次のようになる。
\begin{align}
    \log{p_{\theta} (x_{1:n})}
        & = \log{\prod_{i=1}^{n} p_{\theta} (x_i)} \\
        & = \sum_{i=1}^{n} \log{p_{\theta} (x_i)} \\
        & \ge \sum_{i=1}^{n} L\qty[q(z_i),\ \theta;\ x_i]
\end{align}



\end{document}
