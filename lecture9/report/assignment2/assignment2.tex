\documentclass[dvipdfmx, fleqn]{jsarticle}
\input{/Users/User/Documents/University/report_template/preamble/preamble}
\title{
	統計的機械学習 \\
    第九回 レポート ID: 02
    }
\author{37-196360 \quad 森田涼介}
\begin{document}
\maketitle


以下の,周辺尤度とKL情報量の関係を導出する。
\begin{equation}
    \log{p(x_{1:n} | \eta)} - L\qty[q(z_{1:n}) q(\theta); x_{1:n}] = \mathrm{KL}\qty[q(z_{1:n}) q(\theta) | p(z_{1:n}, \theta | x_{1:n}, \eta)]
\end{equation}

データ\(x_{1:n}\)の生成確率の周辺尤度からその変分下限を得る。
\begin{align}
    \log{p(x_{1:n} | \eta)}
        & = \log{\int p(x_{1:n}, z_{1:n}, \theta | \eta) \dd{z_{1:n}} \dd{\theta}} \\
        & = \log{\int q(z_{1:n}) q(\theta) \frac{p(x_{1:n}, z_{1:n}, \theta | \eta)}{q(z_{1:n}) q(\theta)} \dd{z_{1:n}} \dd{\theta}} \\
        & \ge \int q(z_{1:n}) q(\theta) \log{\frac{p(x_{1:n}, z_{1:n}, \theta | \eta)}{q(z_{1:n}) q(\theta)}} \dd{z_{1:n}} \dd{\theta} \\
        & \equiv L\qty[q(z_{1:n}) q(\theta); x_{1:n}]
\end{align}
また,KL情報量(Kullback-Leibler Divergence)は一般に,
\begin{equation}
    \mathrm{KL}\qty[q(\theta) | p(\theta)] = \int q(\theta) \log{\frac{q(\theta)}{p(\theta)}} \dd\theta
\end{equation}
と表されるから,
\begin{equation}
    \mathrm{KL}\qty[q(z_{1:n}) q(\theta) | p(z_{1:n}, \theta | x_{1:n}, \eta)]
        = \int q(z_{1:n}) q(\theta) \log{\frac{q(z_{1:n}) q(\theta)}{p(z_{1:n}, \theta | x_{1:n}, \eta)}}  \dd{z_{1:n}} \dd{\theta}
\end{equation}
である。
ここで,
\begin{align}
    \frac{p(x_{1:n}, z_{1:n}, \theta | \eta)}{p(z_{1:n}, \theta | x_{1:n}, \eta)}
        & = \frac{p(x_{1:n},\ z_{1:n},\ \theta,\ \eta)}{p(\eta)} \cdot \frac{p(x_{1:n},\ \eta)}{p(x_{1:n},\ z_{1:n},\ \theta,\ \eta)} \\
        & = \frac{p(x_{1:n},\ \eta)}{p(\eta)} \\
        & = p(x_{1:n} | \eta)
\end{align}
であることに注意すると,
変分下限とKL情報量の和は,
\begin{align}
    & L\qty[q(z_{1:n}) q(\theta); x_{1:n}] + \mathrm{KL}\qty[q(z_{1:n}) q(\theta) | p(z_{1:n}, \theta | x_{1:n}, \eta)] \\
    & \ \qquad
        = \int q(z_{1:n}) q(\theta) \log{\frac{p(x_{1:n}, z_{1:n}, \theta | \eta)}{q(z_{1:n}) q(\theta)}} \dd{z_{1:n}} \dd{\theta} + \int q(z_{1:n}) q(\theta) \log{\frac{q(z_{1:n}) q(\theta)}{p(z_{1:n}, \theta | x_{1:n}, \eta)}}  \dd{z_{1:n}} \dd{\theta} \\
    & \ \qquad
        = \int q(z_{1:n}) q(\theta) \log{\frac{p(x_{1:n}, z_{1:n}, \theta | \eta)}{p(z_{1:n}, \theta | x_{1:n}, \eta)}} \dd{z_{1:n}} \dd{\theta} \\
    & \ \qquad
        = \int q(z_{1:n}) q(\theta) \log{p(x_{1:n} | \eta)} \dd{z_{1:n}} \dd{\theta}
\end{align}
となる。
いま,\(q(z_{1:n}),\ q(\theta)\)は確率密度より
\begin{equation}
    \int q(z_{1:n}) \dd{z_{1:n}} = \int q(\theta) \dd{\theta} = 1
\end{equation}
であること,
及び\(\log{p(x_{1:n} | \eta)}\)は\(z_{1:n},\ \theta\)に依存しないことから,
結局,
\begin{equation}
    L\qty[q(z_{1:n}) q(\theta); x_{1:n}] + \mathrm{KL}\qty[q(z_{1:n}) q(\theta) | p(z_{1:n}, \theta | x_{1:n}, \eta)] = \log{p(x_{1:n} | \eta)}
\end{equation}
よって,
\begin{equation}
    \log{p(x_{1:n} | \eta)} - L\qty[q(z_{1:n}) q(\theta); x_{1:n}] = \mathrm{KL}\qty[q(z_{1:n}) q(\theta) | p(z_{1:n}, \theta | x_{1:n}, \eta)]
\end{equation}



\end{document}
