\documentclass[class=jsarticle, crop=false, dvipdfmx, fleqn]{standalone}
\input{../../../preamble}
\begin{document}
\section{}

微分と積分の順序が交換できるとき,
不偏推定量\(\hat{\theta}\)について,次式を示す。
\begin{equation}
	\mathbb{E}_{\bm{x}_1,\ \dots,\ \bm{x}_n \sim q(\cdot;\ \theta)}
			\qty[\hat{\theta} \sum_{i=1}^{n} \eval{\pdv{}{\theta} \log{q(\bm{x}_i;\ \theta)}}_{\theta = \theta^*}]
	= 1
	\label{eq:objective}
\end{equation}

i.i.d.な訓練標本\(\qty{\bm{x}_i}_{i=1}^{n}\)がモデル\(q(\bm{x};\ \theta)\)から生起するとき,
その確率は,
\begin{equation}
	p(\bm{x}_1,\ \dots,\ \bm{x}_n) = \prod_{i=1}^{n} q(\bm{x}_i;\ \theta)
\end{equation}
となる。
このとき,対数尤度とその\(\theta\)による微分は,
\begin{align}
	& L(\theta) = \log{p} = \sum_{i=1}^{n} \log{q(\bm{x}_i;\ \theta)} \\
	& \pdv{L(\theta)}{\theta} = \pdv{}{\theta} (\log{p}) = \sum_{i=1}^{n} \pdv{}{\theta} (\log{q(\bm{x}_i;\ \theta)})
\end{align}
である。

いま,式(\ref{eq:objective})の左辺は次のように変形できる。
\begin{align}
	\mathbb{E}_{\bm{x}_1,\ \dots,\ \bm{x}_n \sim q(\cdot;\ \theta)}
			\qty[\hat{\theta} \sum_{i=1}^{n} \eval{\pdv{}{\theta} \log{q(\bm{x}_i;\ \theta)}}_{\theta = \theta^*}]
		& = \mathbb{E}_{\bm{x} \sim p(\cdot;\ \theta)} \qty[\hat{\theta} \eval{\pdv{}{\theta} (\log{p})}_{\theta = \theta^*}] \\
		& = \int p \hat{\theta} \eval{\pdv{}{\theta} (\log{p})}_{\theta = \theta^*} \ \dd\bm{x}
			\notag \\
		& = \int \hat{\theta} \eval{\qty{p \pdv{}{\theta} (\log{p})}}_{\theta = \theta^*} \ \dd\bm{x} 
			\label{eq:1}
\end{align}

また,\(\mathbb{E}\qty[\hat{\theta}] = \theta^{*}\)の両辺を\(\theta^{*}\)で偏微分することを考える。
微分と積分の順序が交換できるとき,
\begin{align}
	1
	& = \pdv{}{\theta^{*}} \mathbb{E}\qty[\hat{\theta}] \\
	& = \pdv{}{\theta^{*}} \int p \hat{\theta} \ \dd\bm{x} \\
	& = \int \pdv{}{\theta^{*}} \qty(p \hat{\theta}) \ \dd\bm{x} \\
	& = \int \hat{\theta} \pdv{p}{\theta^{*}} \ \dd\bm{x}
\end{align}
となる。いま,
\begin{equation}
	p(\bm{x};\ \theta) \qty(\pdv{}{\theta} \log{p(\bm{x};\ \theta)}) = \pdv{}{\theta} p(\bm{x}; \theta)
\end{equation}
であるから,結局,
\begin{equation}
	\int \hat{\theta} \eval{\qty{p \pdv{}{\theta} (\log{p})}}_{\theta = \theta^*} \ \dd\bm{x} = 1
		\label{eq:2}
\end{equation}
となる。

式(\ref{eq:1}),(\ref{eq:2})から,
\begin{align}
	\mathbb{E}_{\bm{x}_1,\ \dots,\ \bm{x}_n \sim q(\cdot;\ \theta)}
			\qty[\hat{\theta} \sum_{i=1}^{n} \eval{\pdv{}{\theta} \log{q(\bm{x}_i;\ \theta)}}_{\theta = \theta^*}] = \int \hat{\theta} \eval{\qty{p \pdv{}{\theta} (\log{p})}}_{\theta = \theta^*} \ \dd\bm{x} = 1
\end{align}
であることが示された。



\end{document}
