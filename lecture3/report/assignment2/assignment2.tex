\documentclass[class=jsarticle, crop=false, dvipdfmx, fleqn]{standalone}
\input{../../../preamble}
\begin{document}
\section{}



微分と積分の順序が交換できるとき,
フィッシャー情報行列について式(\ref{eq:answer})が成立することを示す。
\begin{align}
	\bm{F}(\bm{\theta})
		& \coloneqq \int \qty(\pdv{}{\bm{\theta}} \log{q(\bm{x};\ \bm{\theta})}) \qty(\pdv{}{\bm{\theta}} \log{q(\bm{x};\ \bm{\theta})})^\mathrm{T} q(\bm{x};\ \bm{\theta}) \ \dd\bm{x} \\
		& = - \int \qty(\pdv{}{\bm{\theta}}{\bm{\theta}^\mathrm{T}} \log{q(\bm{x};\ \bm{\theta})}) q(\bm{x};\ \bm{\theta}) \ \dd\bm{x} \label{eq:answer}
\end{align}

以下では,簡単のため\(q(\bm{x};\ \bm{\theta})\)を\(q\)と記すこととする。
\(q, r \in \mathbb{R}\)について,
\begin{align}
	& \qty(\pdv{}{\bm{\theta}} \log{q}) q = \pdv{}{\bm{\theta}} q \\
	& \qty(\pdv{r}{\bm{\theta}})^\mathrm{T} = \pdv{r}{\bm{\theta}^\mathrm{T}}
\end{align}
が成立することから,
\begin{align}
	\bm{F}(\bm{\theta})
		& \coloneqq \int \qty(\pdv{}{\bm{\theta}} \log{q}) \qty(\pdv{}{\bm{\theta}} \log{q})^\mathrm{T} q \ \dd\bm{x} \\
		& = \int \qty(\pdv{}{\bm{\theta}} \log{q}) q \cdot \qty(\pdv{}{\bm{\theta}} \log{q})^\mathrm{T} \dd\bm{x} \\
		& = \int \pdv{q}{\bm{\theta}} \qty(\pdv{}{\bm{\theta}^\mathrm{T}} \log{q}) \ \dd\bm{x} \label{eq:1}
\end{align}
となる。

いま,
\begin{equation}
	G(\bm{\theta}) \coloneqq \int \pdv{}{\bm{\theta}} \qty(q \pdv{r}{\bm{\theta}^\mathrm{T}}) \ \dd\bm{x}
\end{equation}
なる\(G\)について,
微分と積分の順序が交換できるとき,
\begin{align}
	G(\bm{\theta})
		& = \int \pdv{}{\bm{\theta}} \qty(q \pdv{r}{\bm{\theta}^\mathrm{T}}) \ \dd\bm{x} \\
		& = \pdv{}{\bm{\theta}} \qty(\int q \pdv{r}{\bm{\theta}^\mathrm{T}} \ \dd\bm{x}) \\
		& = \pdv{}{\bm{\theta}} \qty{\int q \dd\bm{x} \cdot \pdv{r}{\bm{\theta}^\mathrm{T}} - \int \qty(\int q \dd\bm{x}) \cdot \pdv{}{\bm{x}} \qty(\pdv{r}{\bm{\theta}^\mathrm{T}}) \ \dd\bm{x}}
\end{align}
が成立する。ここで,
\begin{equation}
	\int q \dd\bm{x} = 1
\end{equation}
から,
\begin{align}
	G(\bm{\theta})
		& = \pdv{}{\bm{\theta}} \qty(\pdv{r}{\bm{\theta}^\mathrm{T}} - \int \pdv{}{\bm{x}} \qty(\pdv{r}{\bm{\theta}^\mathrm{T}}) \ \dd\bm{x}) \\
		& = \pdv{}{\bm{\theta}} \qty(\pdv{r}{\bm{\theta}^\mathrm{T}} - \pdv{r}{\bm{\theta}^\mathrm{T}}) \\
		& = \bm{0}
\end{align}
となる。
また,\(G\)について,次式も成立する。
\begin{align}
	G(\bm{\theta})
		& = \int \pdv{}{\bm{\theta}} \qty(q \pdv{r}{\bm{\theta}^\mathrm{T}}) \ \dd\bm{x} \\
		& = \int \qty(\pdv{q}{\bm{\theta}} \pdv{r}{\bm{\theta}^\mathrm{T}} + q \pdv{r}{\bm{\theta}}{\bm{\theta}^\mathrm{T}})\ \dd\bm{x}
\end{align}
これらのことから,\(r = \log{q}\)とすれば,
\begin{equation}
	\int \pdv{q}{\bm{\theta}} \qty(\pdv{}{\bm{\theta}^\mathrm{T}} \log{q} )\ \dd\bm{x} = - \int q \qty(\pdv{}{\bm{\theta}}{\bm{\theta}^\mathrm{T}} \log{q})\ \dd\bm{x} \label{eq:2}
\end{equation}
が成立することがわかる。

式(\ref{eq:1}),(\ref{eq:2})から,
\begin{align}
	\bm{F}(\bm{\theta}) = - \int \qty(\pdv{}{\bm{\theta}}{\bm{\theta}^\mathrm{T}} \log{q}) q\ \dd\bm{x}
\end{align}
が示される。


\end{document}
