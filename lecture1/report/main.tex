\documentclass[dvipdfmx, fleqn]{jsarticle}
\input{../../preamble}
\title{
	統計的機械学習 \\
	第1回 レポート
	}
\begin{document}
\maketitle


あるコーディネート\(X\)がおしゃれであるという事象\(Y\)を考える。
ここでは簡単に,おしゃれであるか否かを2値問題として考え,
そのコーディネートがおしゃれであるときを\(Y = 1\),
そうでないときを\(Y = 0\)とする。

おしゃれに気を使うコミュニティを考える。
そのコミュニティにおいておしゃれでないものを着てしまったとき(\(Y=0\)),
当然周囲の人からの評価は下がる。
一方おしゃれなコーディネートであれば(\(Y=1\)),当然評価は上がる。
ここで,その評価を数値で表すことを考える。
\(Y=0\)のときの周囲の人からの評価の数値の変動を\(l_0 \ (< 0)\),
\(Y=1\)のときの変動を\(l_1 \ (> 0)\)とする。
このとき,そのコミュニティはおしゃれな人が集まっているので,
おしゃれのハードルは上がっているとすると,
\begin{equation}
	|l_0| > |l_1|	\label{eq:loss_mag_ref}
\end{equation}
となる。

いま,あるコーディネート\(X\)がおしゃれであるかどうかを識別するモデル\(f(X)\)を考え,
その出力を\(\hat{Y}\)と表す。
ここで,
おしゃれであるものをおしゃれでないと誤識別したとき(\(\hat{Y}=0\ |\ Y=1\))の損失\(l_{1, 0}\)は,
おしゃれであるときに得られた評価の上がり幅\(l_1\)に一致し,
また,おしゃれでないものをおしゃれであると誤識別したとき(\(\hat{Y}=1\ |\ Y=0\))の損失\(l_{0, 1}\)は,
おしゃれでないものを着てしまったときの評価の下がり幅の絶対値\(|l_0|\)に一致すると仮定することができる。
このとき,式(\ref{eq:loss_mag_ref})の大小関係より,
\begin{equation}
	l_{0, 1} > l_{1, 0}
\end{equation}
となる。

いま,
おしゃれでないものを着たときの評価の下がり幅が,
おしゃれなものを着たときの評価の上がり幅より10倍大きいと仮定すると,
各損失は次のような値になる。
\begin{align}
	& l_{0, 1} = -l_0 = 10 \\
	& l_{1, 0} = l_1 = 1
\end{align}


\end{document}
