\documentclass[class=jsarticle, crop=false, dvipdfmx, fleqn]{standalone}
\input{../../../preamble}
\begin{document}
\section{}

ガウスカーネルに対するカーネル密度推定法を行う。
バンド幅は尤度交差確認法によって決定する。


結果を以下の表\ref{tab:result}と図\ref{fig:result}に示した。
表\ref{tab:result}より,最も尤度の平均の大きいバンド幅は0.1となった。
なお,プログラムは\pageref{listing:assignment1}ページのListing \ref{listing:assignment1}に示した。


\begin{table}[H]
    \centering
    \caption{バンド幅とそれに対応するLCVの値}
    \begin{tabular}{lrrrr}
        \(h\) & 0.01 & 0.05 & 0.10 & 0.50 \\
        LCV & \(-6344\) & \(-4009\) & \(-3938\) & \(-4219\)
    \end{tabular}
    \label{tab:result}
\end{table}

\begin{figure}[H]
    \centering
    \includegraphics[clip, width=18cm]{../figures/assignment1_result_global}
    \caption{各バンド幅に対するヒストグラムと確率密度}
    \label{fig:result}
\end{figure}


\end{document}
