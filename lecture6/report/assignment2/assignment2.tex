\documentclass[class=jsarticle, crop=false, dvipdfmx, fleqn]{standalone}
%% preamble for Numerical-structure-analysis report

\input{/Users/User/Documents/Project/TeX/preamble/mypreamble}

%% titles
\title{統計的機械学習 レポート}
\author{37-196360 \quad 森田涼介}


%% setting for listings
\newtcbinputlisting[auto counter]{\reportlisting}[3][]{%
	listing file = {#3},
	listing options = {language=python, style=tcblatex, numbers=left, numberstyle=\tiny},
	listing only,
	breakable,
	toprule at break = 0mm,
	bottomrule at break = 0mm,
	left = 6mm,
	sharp corners,
	drop shadow,
	title = Listings \thetcbcounter : \texttt{#2},
	label = #1,
	}



%% title format
\usepackage{titlesec}
\titleformat{\section}{\LARGE}{宿題\thesection}{0zw}{}
\newcommand{\sectionbreak}{\clearpage}
\titleformat{\subsection}{\Large}{\Alph{subsection})}{0zw}{}

\begin{document}
\section{}

最近傍識別器によって0--9までの10クラスの手書き文字認識を行う。
近傍数\(k\)は識別誤差に関する交差確認により決定する。

以下に結果を示す。
近傍数\(k\)の候補として,1--10を考えた。
各\(k\)についての,
交差確認法で求めた正解数の平均値を表\ref{tab:k_result}に示す。
これより,最も高い正解数を与える\(k\)を選ぶと,
\(k = 3\)となる。
\(k = 3\)に対する混同行列は表\ref{tab:confusion_matrix}のようになり,
また,各カテゴリごとの正解率等は表\ref{tab:result}のようになった。

プログラムは\pageref{listing:assignment2}ページのListing \ref{listing:assignment2}に示した。


\begin{table}[H]
	\centering
	\caption{各\(k\)についての,交差確認法で求めた正解数の平均値}
	\begin{tabular}{cccccccccc}
		1 & 2 & 3 & 4 & 5 & 6 & 7 & 8 & 9 & 10 \\ \hline
		484.7 & 483.2 & 485.1 & 484.8 & 483.4 & 482.7 & 481.6 & 481.2 & 480.0 & 479.7
	\end{tabular}
	\label{tab:k_result}
\end{table}

\begin{table}[H]
	\centering
	\caption{混同行列}
	\begin{tabular}{|c||cccccccccc|} \hline
			& 0 & 1 & 2 & 3 & 4 & 5 & 6 & 7 & 8 & 9 \\ \hline\hline
		0     & 198 & 0 & 1 & 0 & 0 & 0 & 1 & 0 & 0 & 0 \\
		1     & 0 & 199 & 1 & 0 & 0 & 0 & 0 & 0 & 0 & 0 \\
		2     & 0 & 0 & 195 & 0 & 0 & 0 & 0 & 2 & 3 & 0 \\
		3     & 0 & 0 & 0 & 190 & 0 & 4 & 0 & 1 & 4 & 1 \\
		4     & 0 & 1 & 0 & 0 & 189 & 0 & 3 & 0 & 0 & 7 \\
		5     & 2 & 0 & 3 & 3 & 1 & 186 & 0 & 0 & 1 & 4 \\
		6     & 1 & 0 & 2 & 0 & 0 & 0 & 197 & 0 & 0 & 0 \\
		7     & 0 & 1 & 0 & 0 & 4 & 0 & 0 & 191 & 0 & 4 \\
		8     & 2 & 0 & 2 & 3 & 0 & 3 & 0 & 0 & 187 & 3 \\
		9     & 0 & 0 & 0 & 0 & 2 & 0 & 0 & 0 & 0 & 198 \\
		\hline
	\end{tabular}
	\label{tab:confusion_matrix}
\end{table}

\begin{table}[H]
	\centering
	\caption{各カテゴリごとの結果}
	\begin{tabular}{crrr}
		Category & {\#}Data & {\#}Correct & Accuracy \\ \hline
		0 & 200 & 198 & 0.990 \\
		1 & 200 & 199 & 0.995 \\
		2 & 200 & 195 & 0.975 \\
		3 & 200 & 190 & 0.950 \\
		4 & 200 & 189 & 0.945 \\
		5 & 200 & 186 & 0.930 \\
		6 & 200 & 197 & 0.985 \\
		7 & 200 & 191 & 0.955 \\
		8 & 200 & 187 & 0.935 \\
		9 & 200 & 198 & 0.990 \\
		All & 2,000 & 1,930 & 0.965
	\end{tabular}
	\label{tab:result}
\end{table}


\end{document}
