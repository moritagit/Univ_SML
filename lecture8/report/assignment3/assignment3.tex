\documentclass[dvipdfmx, fleqn]{jsarticle}
%% preamble for Numerical-structure-analysis report

\input{/Users/User/Documents/Project/TeX/preamble/mypreamble}

%% titles
\title{統計的機械学習 レポート}
\author{37-196360 \quad 森田涼介}


%% setting for listings
\newtcbinputlisting[auto counter]{\reportlisting}[3][]{%
	listing file = {#3},
	listing options = {language=python, style=tcblatex, numbers=left, numberstyle=\tiny},
	listing only,
	breakable,
	toprule at break = 0mm,
	bottomrule at break = 0mm,
	left = 6mm,
	sharp corners,
	drop shadow,
	title = Listings \thetcbcounter : \texttt{#2},
	label = #1,
	}



%% title format
\usepackage{titlesec}
\titleformat{\section}{\LARGE}{宿題\thesection}{0zw}{}
\newcommand{\sectionbreak}{\clearpage}
\titleformat{\subsection}{\Large}{\Alph{subsection})}{0zw}{}

\title{
	統計的機械学習 \\
    第八回 レポート ID: 03
    }
\author{37-196360 \quad 森田涼介}
\begin{document}
\maketitle

ベイズ推定の逐次合理性を示す。
データ\(x_{1:n} = (x_1,\ \cdots,\ x_n)\)について,
条件付き独立性を仮定すると,尤度は,
\begin{equation}
    p(x_{1:n}|\theta) = \prod_{i=1}^{n} p(x_i | \theta)
    \label{eq:likelihood}
\end{equation}
また,De Finettiの定理より,確率変数の列\(x_{1:n}\)が交換可能であるとき,
任意\(n\)に対して次が成立する。
\begin{equation}
    p(x_{1:n}) = \int \prod_{i=1}^{n} p(x_i | \theta) p(\theta) \dd\theta
    \label{eq:marginalization}
\end{equation}
これより,事前分布を\(p(\theta)\)とすると,事後分布は,
\begin{equation}
    p(\theta|x_{1:n}) = \frac{p(x_{1:n} | \theta) p(\theta)}{p(x_{1:n})}
    \label{eq:posterior}
\end{equation}
となる。
式(\ref{eq:likelihood})を用いると,
\begin{align*}
    p(x_{1:n} | \theta) p(\theta)
        & = \prod_{i=1}^{n} p(x_i | \theta) p(\theta) \\
        & = p(x_n | \theta) \prod_{i=1}^{n-1} p(x_i | \theta) p(\theta) \\
        & = p(x_n | \theta) p(x_{1:n-1} | \theta) p(\theta)
\end{align*}
ベイズの定理より,
\begin{equation*}
    p(x_{1:n-1} | \theta) p(\theta) = p(\theta | x_{1:n-1}) p(x_{1:n-1})
\end{equation*}
が成立するので,結局,
\begin{equation}
    p(x_{1:n} | \theta) p(\theta) = p(x_n | \theta) p(\theta | x_{1:n-1}) p(x_{1:n-1})
    \label{eq:sequentialization}
\end{equation}
式(\ref{eq:marginalization}),(\ref{eq:posterior}),(\ref{eq:sequentialization})から,
\begin{align*}
    p(\theta|x_{1:n})
        & = \frac{p(x_{1:n} | \theta) p(\theta)}{p(x_{1:n})} \\
        & = \frac{p(x_n | \theta) p(\theta | x_{1:n-1}) p(x_{1:n-1})}{\int p(x_n | \theta) p(\theta | x_{1:n-1}) p(x_{1:n-1}) \dd\theta} \\
        & = \frac{p(x_n | \theta) p(\theta | x_{1:n-1})}{\int p(x_n | \theta) p(\theta | x_{1:n-1}) \dd\theta}
\end{align*}
これより,\(p(\theta | x_{1:n-1})\)を事前分布とすると,
事後分布は,
\begin{equation}
    p(\theta|x_{1:n}) = \frac{p(x_n | \theta) p(\theta | x_{1:n-1})}{\int p(x_n | \theta) p(\theta | x_{1:n-1}) \dd\theta}
\end{equation}
と表すことができ,逐次性を持つことが示された。


\end{document}
