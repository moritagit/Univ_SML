\documentclass[dvipdfmx, fleqn]{jsarticle}
%% preamble for Numerical-structure-analysis report

\input{/Users/User/Documents/Project/TeX/preamble/mypreamble}

%% titles
\title{統計的機械学習 レポート}
\author{37-196360 \quad 森田涼介}


%% setting for listings
\newtcbinputlisting[auto counter]{\reportlisting}[3][]{%
	listing file = {#3},
	listing options = {language=python, style=tcblatex, numbers=left, numberstyle=\tiny},
	listing only,
	breakable,
	toprule at break = 0mm,
	bottomrule at break = 0mm,
	left = 6mm,
	sharp corners,
	drop shadow,
	title = Listings \thetcbcounter : \texttt{#2},
	label = #1,
	}



%% title format
\usepackage{titlesec}
\titleformat{\section}{\LARGE}{宿題\thesection}{0zw}{}
\newcommand{\sectionbreak}{\clearpage}
\titleformat{\subsection}{\Large}{\Alph{subsection})}{0zw}{}

\title{
	統計的機械学習 \\
    第八回 レポート ID: 02
    }
\author{37-196360 \quad 森田涼介}
\begin{document}
\maketitle

ある検査方法を分析するために被験者を集めることにした。
陽性であった被験者を5人集めるために全部で20人の被験者を必要とした。
少なくともあと2人陽性の被験者のデータを取るために,
何人の被験者を集めれば良いかを考える。

事前分布を,\(a,\ b\)をパラメータとするベータ分布,
尤度を負の二項分布とする。
陽性となる確率を\(\pi\)とするBernoulli分布に従う\(n\)回の独立した試行において,
陽性となる回数が\(k\)となるまでに陰性となった回数を\(m\)とすると,
\begin{align}
    & p(\pi) = \mathrm{Beta}(\pi|a, b)
        = \frac{\Gamma(a+b)}{\Gamma(a) \Gamma(b)} \pi^{a-1} (1-\pi)^{b-1} \\
    & p(\mathrm{data}|\pi)
        = \mathrm{NB}(m|\pi)
        = \frac{n!}{k! (m+1)!} \pi^k (1-\pi)^m
\end{align}
\(p(\mathrm{data})\)は定数であることに注意すると,
\(\pi\)の事後分布の確率密度関数は,
\begin{align}
    p(\pi | \mathrm{data})
        & = \frac{p(\mathrm{data} | \pi) p(\pi)}{p(\mathrm{data})} \\
        & \propto p(\mathrm{data} | \pi) p(\pi) \\
        & = \frac{\Gamma(a+b)}{\Gamma(a) \Gamma(b)} \pi^{a-1} (1-\pi)^{b-1} \cdot \frac{n!}{k! (m+1)!} \pi^k (1-\pi)^m \\
        & \propto \pi^{a + k - 1} (1-\pi)^{b + m - 1}
\end{align}
これを正規化すると,結局,事後分布は,
\begin{equation}
    p(\pi | \mathrm{data}) = \mathrm{Beta}(\pi | a + k,\ b + m)
\end{equation}
となる。
また,\(l\)回の成功が得られるまでの失敗の数\(x\)の期待値は,
\begin{align}
    & E_{\mathrm{NB}(x|\pi)} \qty[x] = l \frac{1 - \pi}{\pi}
    \label{eq:bound} \\
    & E \qty[x|\mathrm{data}]
        = \int_0^1 E_{\mathrm{NB}(x|\pi)} \qty[x] p(\pi | \mathrm{data}) \dd\pi
        = \int_0^1 l \frac{\Gamma(a+b+k+m)}{\Gamma(a+k) \Gamma(b+m)} \pi^{a+k-2} (1-\pi)^{b+m} \dd\pi
    \label{eq:mean}
\end{align}
となる。

\(a = b = 1,\ m = 15,\ k = 5\)のときを考える。
あと\(l = 2\)人陽性の人を集めるために平均的に必要な人数は,
\(l + E \qty[x|\mathrm{data}]\)である。
式(\ref{eq:mean})を用いて,数値計算によりこれを求めると,
\begin{equation}
    l + E \qty[x|\mathrm{data}] = 2 + 6.4 = 8.4
\end{equation}
となる。

また,多く見積もることを考えると,\(\pi\)が比較的小さい値を取るとすればよい。
例えば,\(\pi \le 0.1\)となるときを考えると,その確率は,
\begin{equation}
    p(\pi \le 0 | \mathrm{data}) = \int_0^{0.1} p(\pi | \mathrm{data}) \dd\pi = 0.0145
\end{equation}
\(\pi = 0.1\)のとき,
あと\(l = 2\)人陽性の人を集めるために必要な人数は,
\begin{equation}
    l + E_{\mathrm{NB}(x|\pi)} \qty[x] = \frac{l}{\pi} = 20
\end{equation}
となる。



\subsection*{プログラム}

\reportlisting[listing:assignment2]{assignment2.py}{../program/assignment2.py}


\end{document}
